\documentclass[11pt]{article}
\usepackage[margin=2.54cm]{geometry}
\usepackage[dvipsnames,usenames]{color}
\usepackage{amsfonts,amsmath,amssymb,amsthm,mathtools}
\usepackage{paralist}
\usepackage{algorithmic,algorithm}
\usepackage{bm}
\usepackage{xspace}
\usepackage{centernot}
\usepackage{fancybox}
\usepackage{framed}

%\usepackage{mathabx}
\usepackage[pagebackref,letterpaper=true,colorlinks=true,pdfpagemode=none,urlcolor=blue,linkcolor=blue,citecolor=BrickRed,pdfstartview=FitH]{hyperref}
\usepackage{xspace,prettyref}
\usepackage{color}
\usepackage{graphics}
%\usepackage{MnSymbol}
%% For hyperlinks. Should always be the last package.
%\usepackage[colorlinks,urlcolor=blue,citecolor=blue,linkcolor=blue]{hyperref}
\let\pref=\prettyref
\newcommand{\mathcalavehyperref}[2]{\texorpdfstring{\hyperref[#1]{#2}}{#2}}
\newcommand{\comment}[1]{ {\color{BrickRed} \footnotesize[#1]}\marginpar{\footnotesize\textbf{\color{red} To Do!}}}
\newcommand{\ignore}[1]{}


\newtheorem{theorem}{Theorem}
\newtheorem{lemmata}[theorem]{Lemmata}
\newtheorem{lemma}[theorem]{Lemma}
\newtheorem{claim}[theorem]{Claim}
\newtheorem{subclaim}{Subclaim}
\newtheorem{proposition}[theorem]{Proposition}
\newtheorem{corollary}[theorem]{Corollary}
\newtheorem{fact}[theorem]{Fact}
\newtheorem{conjecture}[theorem]{Conjecture}
\newtheorem{question}[theorem]{Question}
\newtheorem{example}[theorem]{Example}
\newtheorem{definition}[theorem]{Definition}
\theoremstyle{definition}
\newtheorem{remark}{Remark}
\newtheorem{observation}{Observation}


\newcommand{\EPSfigure}[5]{

        % #1 = File name and other arguments to \psfig macro
        % #2 = Caption Text
        % #3 = Positioning Letters: h, t|b|p
        % #4 = {*} causes double column figure, {} for single
        % #5 = Label to apply to figure

        \begin{figure#4}[#3]
                \centering
                \ \psfig{file=#1}
%               \label{#5}
%
                \ \caption{{\em #2}\label{#5}}\hfill\break

        \end{figure#4}
}

\newcommand{\innbd}{\Gamma^-}
\newcommand{\nbd}{\Gamma}
\newcommand{\outnbd}{\Gamma^+}

\newcommand{\bT}{{\bf T}}
\newcommand{\cS}{{\cal S}}
\newcommand{\cB}{{\cal B}}
\newcommand{\cC}{{\cal C}}
\newcommand{\cD}{{\cal D}}
\newcommand{\cE}{{\cal E}}
\newcommand{\cF}{{\cal F}}
\newcommand{\cG}{{\cal G}}
\newcommand{\cH}{{\cal H}}
\newcommand{\cL}{{\cal L}}
\newcommand{\cP}{{\cal P}}
\newcommand{\cV}{{\cal V}}

\newcommand{\LL}{\mathbb{L}}
\newcommand{\MM}{\mathbb{M}}
\newcommand{\NN}{\mathbb{N}}
\newcommand{\RR}{\mathbb{R}}
\newcommand{\SSS}{\mathbb{S}}



\newcommand{\bmu}{\overline{\mu}}
\newcommand{\poly}{\textrm{poly}}
\newcommand{\mymod}{\textrm{mod} \ }
\newcommand{\otilde}{\widetilde{O}}
%\newcommand{\qed}{\hfill $\Box$}
\newcommand{\eps}{\varepsilon}
\newcommand{\fu}{\varphi}
\newcommand{\restr}[1]{{|_{{\textstyle #1}}}}
\newcommand{\proj}{\mbox{\rm proj}}
\newcommand{\vol}{\mbox{\tt vol}\,}
\newcommand{\area}{\mbox{\tt area}\,}
\newcommand{\conv}{\mbox{\tt conv}\,}
\newcommand{\diam}{\hbox{\tt diam}\,}
\newcommand{\hdisc}{\mbox{\rm herdisc}}
\newcommand{\ldisc}{\mbox{\rm lindisc}}
\newcommand{\EX}{\hbox{\bf E}}
\newcommand{\prob}{{\rm Prob}}
\newcommand{\proofend}{{\medskip\medskip}}
%\newcommand{\proof}{{\noindent\bf Proof: }}
%\newcommand{\reals}{{\rm I\!\hspace{-0.025em} R}}
%\newcommand{\dist}{\hbox{dist}}
\newcommand{\defeq}{\mbox{\,$\stackrel{\rm def}{=}$\,}}
\newcommand{\boxalg}[1]
{\begin{center}\fbox{\parbox{\columnwidth}{\tt
\begin{tabbing}
\=mm\=mm\=mm\=mm\=mm\=mm\=mm\=mm\=mm\kill
#1
\end{tabbing} } } \end{center} }

%% HYPER-LINKED REFERENCES
\newcommand{\Sec}[1]{\hyperref[sec:#1]{\S\ref*{sec:#1}}} %section
\newcommand{\Eqn}[1]{\hyperref[eqn:#1]{(\ref*{eqn:#1})}} %equation
\newcommand{\Clm}[1]{\hyperref[clm:#1]{Claim~\ref*{clm:#1}}} %claim
\newcommand{\Fig}[1]{\hyperref[fig:#1]{Figure~\ref*{fig:#1}}} %figure
\newcommand{\Tab}[1]{\hyperref[tab:#1]{Table~\ref*{tab:#1}}} %table
\newcommand{\Thm}[1]{\hyperref[thm:#1]{Theorem~\ref*{thm:#1}}} %theorem
\newcommand{\Lem}[1]{\hyperref[lem:#1]{Lemma~\ref*{lem:#1}}} %lemma
\newcommand{\Prop}[1]{\hyperref[prop:#1]{Proposition~\ref*{prop:#1}}} %property
\newcommand{\Cor}[1]{\hyperref[cor:#1]{Corollary~\ref*{cor:#1}}} %corollary
\newcommand{\Def}[1]{\hyperref[def:#1]{Definition~\ref*{def:#1}}} %definition
\newcommand{\Alg}[1]{\hyperref[alg:#1]{Algorithm~\ref*{alg:#1}}} %algorithm
\newcommand{\Ex}[1]{\hyperref[ex:#1]{Example~\ref*{ex:#1}}} %example

%% Comments to ourselves
\newcommand{\Reminder}[1]{{\color{red}#1}}
\newcommand{\Sesh}[1]{\Reminder{Sesh interjects: #1}}

%%Ben Added
\DefineNamedColor{named}{RedViolet} {cmyk}{0.07,0.90,0,0.34}
\providecommand{\AlgorithmI}[1]{{\textcolor[named]{RedViolet}{\texttt{\bf{#1}}}}}
\providecommand{\Algorithm}[1]{{\AlgorithmI{#1}\index{algorithm!#1@{\AlgorithmI{#1}}}}}
\newcommand{\zip}{\Algorithm{ZIP}}
\newcommand{\link}{\Algorithm{LINK}}
\newcommand{\sizes}{\Algorithm{SIZES}}
\newcommand{\AMT}{\ensuremath{\mathsf{AMT}}\xspace}
\newcommand{\MT}{\ensuremath{\mathsf{MT}}\xspace}
\newcommand{\BCS}{\ensuremath{\mathsf{BCS}}\xspace}
\newcommand{\sub}{\mathsf{sub}}
\newcommand{\Merge}{\mathsf{Merge}}
\newcommand{\stack}{\mathsf{S}}
\newcommand{\AQ}{\mathsf{AQ}}

\newcommand{\build}{{\tt build}}
\newcommand{\col}{col}
\newcommand{\cut}{{\tt cut}}
\newcommand{\cont}{\psi}
\newcommand{\h}{high}
\newcommand{\lift}{{\tt lift}}
\newcommand{\mcol}{mcol}
\newcommand{\onestep}{{\tt piece}}
\newcommand{\paint}{{\tt paint}}
\newcommand{\rain}{{\tt rain}}
\newcommand{\reeb}{R}
\newcommand{\rep}{rep}
\newcommand{\st}{K}
\newcommand{\surgery}{{\tt surgery}}
\newcommand{\touch}{T}
\newcommand{\update}{{\tt update}}
\newcommand{\wet}{{\tt wet}}

\newcommand{\MTAlg}{\Algorithm{MTAlg}\xspace}
\newcommand{\Init}{\Algorithm{Init}\xspace}

\newcommand{\CodeComment}[1]{\textcolor{blue}{\texttt{#1}}}



\author{}

\title{Follow the Flow of Rain and Paint: \break An Output Sensitive Contour Tree Algorithm}
\date{}


\begin{document}

\maketitle



\section{Preliminaries}
Consider a piecewise-linear (PL) $2$-manifold $\MM$ with boundaries.
This is represented by a triangulation (given as a DCEL) where edges
can be oriented and the boundary of a face is always counterclockwise. Some faces will \emph{not}
be triangular, and each of these corresponds to a \emph{boundary face}.
Let $G(\MM) = (V,E)$ denote this triangulation. We assume this triangulation
has bounded degree.
We will assume that $\MM$ is embedding in $\RR^d$.
The manifold within each face is just linear.
We define a \emph{height function} on $\MM$, $f:\MM \mapsto \RR$. For each point in $x \in \MM$, let $f(x)$ simply be the value of $x_1$.
We distinguish between vertices and points in $\MM$. A point simply denotes any $x \in \MM$. A vertex is a point
corresponding to some vertex of the triangulation $G(\MM)$.

We define the size of the manifold $\MM$, denoted $|\MM|$, to be the number of faces, edges, and vertices.  
Let $n$ denote the number of vertices.  
Since we are working with $2$-manifolds, the size of the manifold is $\Theta(n)$.

For smooth manifolds the ``status" of an internal point is decided by the gradient at that point. 
Points of zero gradient are deemed \emph{critical};
all other points are regular.
For PL manifolds, we need a discrete version of this definition.
First, all non-vertex points are considered regular.
For vertex $v$, we use $\nbd(v)$ to denote the neighborhood of $v$ in $G(\MM)$. Let $\outnbd(v) := \{ w | f(v) > f(w), w \in \Gamma(v)\}$
and $\innbd(v) := \{w | f(v) < f(w), w \in \Gamma(v)\}$. 
%It is convenient to define $D(\MM)$, a directed variant of $G(\MM)$,
%where edges are directed from higher to lower $f$-value. So $\outnbd(v)$ and $\innbd(v)$ are exactly the out- and in-neighborhoods
%of $v$. A vertex in $v \in V$ can be classified in four types.
\smallskip
\begin{asparaenum}
	\item Regular: Both $\outnbd(v)$ and $\innbd(v)$ are non-empty and are contiguous in $\nbd(v)$.
	\item Maxima: $\nbd(v) = \outnbd(v)$.
	\item Minima: $\nbd(v) = \innbd(v)$.
	\item Saddle: Both $\outnbd(v)$ and $\innbd(v)$ are not contiguous in $\nbd(v)$.
\end{asparaenum}
\smallskip
%Because $\MM$ is PL, the gradient at all non-vertex points of $\MM$ is zero.
%So all non-vertex points are regular. 

We will assume that $f$ is Morse, so the function value on all critical points is distinct. 
We will also assume the stronger condition that all internal vertices have distinct function values.
Note that assuming $f$ is Morse implies that $\MM$ has no multi-saddles 
(i.e. $\outnbd(v)$ and $\innbd(v)$ can each have at most two non-contiguous components).
A value $t$ is \emph{non-critical} if there exists no critical point $x$ such that $f(x) = t$.

Furthermore, the boundary vertices have special properties, encapsulated in the following definition.

\begin{definition} \label{def:bound} A manifold is \emph{boundary critical} if the following holds.
Consider a boundary face $F$. All vertices in $F$ have the same function value. Furthermore, all
internal neighbors of $F$ either have function values strictly greater (or strictly less)
than this value.
\end{definition}

In other words, the vertices in a boundary face collectively behave like a maximum or minimum. 
Abusing notation, the term ``critical point", ``maxima", and ``minima" will also refer to boundaries.
Critical values include these boundary values.

\begin{definition} \label{def:desc} A \emph{descending path} in $\MM$ is a continuous function $\lambda:[0,1] \mapsto \MM$
such that for all $a < b$, $f(\lambda(a)) \geq f(\lambda(b))$. A \emph{PL descending path} from $x$ to $y$ is a sequence
of points $x = s_0, s_1, s_2, \ldots, s_k = y$ with the following properties: for all $s_i$ but the first and last,
$s_i$ lies on an edge of $\MM$. $s_i$ and $s_{i+1}$ lie in the same face of $\MM$. 
For $i < j$, $f(s_i) \geq f(s_j)$.
\end{definition}

Since $\MM$ is PL, if there is a descending path from $x$ to $y$ then it has PL representation. So we will always refer to
PL descending paths.

\begin{definition} \label{def:cont} A \emph{contour} in $\MM$ is the image of a continuous and injective function $\phi:\SSS^1 \mapsto \MM$ such that $\forall a, b \in \SSS^1$, $f(\phi(a)) = f(\phi(b))$. This is called a \emph{$t$-contour} if $f(\phi(a)) = t$.
%A contour is \emph{simple} if $\forall a, b \in \SSS^1$, $\lambda(a) \neq \lambda(b)$. 
\end{definition}

Note that just as in the case of paths, contours are also PL and have a corresponding discrete representation.

\begin{remark}
For contours which pass through critical points, our definition of a contour differs from the ``standard" definition (i.e. contours which are the result of a horizontal cutting).  

In particular, the injective requirement does not allow contours through maxima or minima.  We can add these single point contours at the maxima and minima to our set of contours (though this is not really necessary).
\end{remark}

\begin{lemma} \label{lem:cont} If two distinct contours intersect, then their intersection is a saddle point.
\end{lemma} 

\begin{proof}
Consider any (closed) triangular face $f$, interior to the $2$-manifold.  The set of points on this face at a given height $h$ forms a closed line segment, call it $l_f$.  Let $\phi$ be any contour that intersects some point in the interior of this face at height $h$.  Then $\phi$ must contain all of $l_f$, since otherwise continuity or simplicity would be violated.  

So consider two distinct intersecting contours $\phi_1$ and $\phi_2$.  Since they are distinct, there must be a point on the boundary of their intersection, i.e. any (arbitrarily small) ball centered at this point contains contour points both inside and outside the intersection of the contours.  Call this point $p$.

If $p$ lies in the interior of a face $f$ then both $\phi_1$ and $\phi_2$ must contain the entire above mentioned closed lines segment $l_f$, and hence $p$ was not on the intersection boundary. 
Similarly suppose that $p$ is in the interior of an edge (clearly such an edge would need to be interior to the manifold for $\phi_1$ and $\phi_2$ to be contours).  Since this is a $2$-manifold each edge is adjacent to exactly two faces.  Let $l_a$ and $l_b$ be the above described closed line segments on each face.  Then again $\phi_1$ and $\phi_2$ contain all of $l_a$ and $l_b$, and $p$ was not on the intersection boundary.

So $p$ must be a vertex $v$ of the manifold.  Let $\phi$ be any contour through $v$.  There must be two distinct faces $f_a$ and $f_b$ adjacent to $v$ which $\phi$ intersects.  Any face adjacent to $v$ has exactly two bounding edges adjacent to $v$.  Since faces are linear interpolates of the vertices, and since $\phi$ intersects $f_a$ and $f_b$, for either face one of these two edges must be increasing from $v$ and the other decreasing from $v$.  Call such a face a middle face.

Now if $v$ is on the boundary of the intersection of $\phi_1$ and $\phi_2$ then we need at least 3 middle faces adjacent to $v$ (since otherwise it again is like the case we $p$ was interior to an edge).  However, the only vertices the have at least 3 such faces are by definition saddle points.  As no two vertices are at the same height, this saddle point can be the only point of intersection.
\end{proof}

%\begin{proof} Consider the intersection point $x$. It cannot be internal to a face, otherwise $\MM$ self-intersects.
%By the distinctness of the contours, there must be $4$ faces incident to $x$, where each face
%has vertex with a larger function value.
%It cannot be internal to an edge, otherwise this edge would participate in $4$ faces. So it must be a vertex. 
%These $4$ faces lead to at least $2$ neighbors of $x$ with higher function value. One can also show the existence of $2$
%interleaved neighbors with lower function value. So $x$ is critical.
%\end{proof}

\begin{corollary}
There is a unique contour containing a regular point.
\end{corollary}

\begin{definition} \label{def:p-cont}
For a regular point $x$, $\cont(x)$ denotes the unique contour containing $x$.
\end{definition}

%By the Morse property of distinct critical values, we get the following important lemma.

\begin{corollary} \label{cor:cont} Let $t$ be a non-critical value. The set of $t$-contours is a set
of disjoint PL contours.
\end{corollary}

We can use contours to classify saddle points. For a saddle point, we call two increasing (or decreasing) neighbors \emph{opposite}
if they lie in disjoint parts of $\Gamma^+(w)$ (or $\Gamma^-(w)$).

In the following definition and, unless otherwise stated, throughout the rest of this writeup $\eps$ is some infinitesimally small amount
(in particular smaller than the minimum height difference between vertices in the manifold).

\begin{definition} \label{def:merge} Consider two increasing (resp. decreasing) opposite edges incident to a saddle point $x$,
and let $y$ and $z$ be points on these edges such that $f(y) = f(z) = f(x) + \eps$ (resp. $=f(x) - \eps$).
If $\cont(y)$ and $\cont(z)$ are disjoint, then $x$ is called a \emph{merge} (resp. \emph{split}) vertex.
\end{definition}

As $f$ is a Morse function over a $2$-manifold, we have the following.

\begin{claim} \label{clm:saddle} 
  \begin{compactenum}[(1)]
    \item Every saddle is either a merge or a split, but not both.
    \item Every saddle has exactly two distinct contours which contain it.
  \end{compactenum}
\end{claim}


\subsection{The L-homotopy} \label{sec:l-hom}
We need a specific notion of homotopy between contours. The L below stands for ``level".

\begin{definition} \label{def:cont-hom} Two contours $\phi_0, \phi_1$ are \emph{L-homotopic}
if: there exists a continuous function $H:\SSS^1 \times [0,1] \mapsto \MM$
such that $H(\SSS^1,0) = \phi_0$, $H(\SSS^1,1) = \phi_1$, and $\forall a \in [0,1]$,
$H(\SSS^1,a)$ is a contour.
\end{definition}

\begin{claim} \label{clm:non-hom} Consider a saddle point $x$. Take an increasing edge incident to $x$
and let $y$ be a point on this edge within an $\eps$-ball of $x$. Similarly, take a decreasing edge
and a point $z$ on this edge. $\psi(y)$ and $\psi(z)$ are not L-homotopic.
\end{claim}

\begin{proof}
Without loss of generality assume that $x$ is a merge vertex.  By \Clm{saddle} there are exactly two distinct contours, 
$\phi_1$ and $\phi_2$ through $x$.  Let $E_1$ and $E_2$ be the sets of manifold edges that are not adjacent to $x$ and that $\phi_1$ and $\phi_2$ intersect, respectively.
Observe that as contours are by definition non-trivial, both $E_1$ and $E_2$ must non-empty.  Moreover since $\phi_1$ and $\phi_2$ only intersect at $x$, $E_1\cap E_2 = \emptyset$

As $x$ is a merge vertex, $\psi(y)$ cannot intersect all edges in $E_1 \cup E_2$.
On the other hand since $x$ cannot be both a merge and a split by \Clm{saddle}, $\psi(z)$ must intersect all of $E_1 \cup E_2$.  As $x$ is the only vertex in between $z$ and $y$ 
and no edge in $E_1\cup E_2$ is adjacent to $x$, it is not hard to see that by the continuity requirement $\psi(z)$ cannot be L-homotopic to $\psi(y)$.
%
%
% Without loss of generality assume that $x$ is a merge vertex.  By \ref{clm:saddle} there are exactly two distinct contours, 
% $\phi_1$ and $\phi_2$ through $x$.  Observe that by continuity of L-homotopy, if $\psi(z)$ is $L$-homotopic to $\psi(y)$, 
% then $\psi(z)$ must L-homotopic to either $\phi_1$ or $\phi_2$, as these are the only two possible contours at the height of $x$ 
% (in the connected component of the manifold in between $y$ and $z$ and containing $x$).
% We now show that $\psi(z)$ is not L-homotopic to either $\phi_1$ or $\phi_2$.
% 
% First observe that since contours are non-trivial (except at extrema), by continuity $\psi(z)$ cannot be 
% simultaneously L-homotopic to both contours through $x$.  So suppose $\psi(z)$ is L-homotopic to one of the contours, say $\phi_1$.
% 
% As $\eps$ is smaller than the minimum height difference of two vertices in the manifold, 
% and $z$ is $\eps$ lower than $x$, there are no vertices in between $z$ and $x$.  In particular, whatever edges $\phi_1$ and $\phi_2$ intersect (in their interier), 
% $\psi(z)$ must also intersect.
\end{proof}

\begin{definition} \label{def:hom-mono} Two contours $\phi_0, \phi_1$ are \emph{monotonically L-homotopic}
if the $L$-homotopy $H$ has the following property. Consider function $\alpha:[0,1] \mapsto \RR$
where $\alpha(a) = f^{-1}(H(\SSS^1,a))$. The function $\alpha$ is strictly increasing or strictly decreasing.
\end{definition}

\begin{claim} \label{clm:mono} If two contours are L-homotopic, then they are monotonically L-homotopic.
\end{claim}

\begin{proof} Consider some L-homotopy $H$. We can assume wlog that for all $a \neq b$, $H(\SSS^1,a) \neq H(\SSS^1,b)$.
(Otherwise, one could simplify $H$ and ``shortcut" between $a$ and $b$.)
By continuity of $H$, the function $\alpha$ is continuous. Hence, if it is not strictly
increasing or decreasing, $\alpha$ attains a local maximum at some $a \in (0,1)$. So there exist $a_1 = a - \eps_1$
and $a_2 = a + \eps_2$ such that $\alpha(a_1) = \alpha(a_2)$.

By continuity, $\eps_1$ and $\eps_2$ can be chosen small enough such that there are no vertices 
in between $f^{-1}(H(\SSS^1,a_1))$ and the local maximum. 
Now the local maximum at $a$ may contain a vertex, but by \Clm{non-hom} this vertex cannot be a saddle.
Therefore $H(\SSS^1,a_1)$ and $H(\SSS^1,a_2)$ must intersect the same set of edges and therefore cannot be distinct.
\end{proof}

\begin{claim} \label{clm:reg} Consider a contour $\phi$ containing only regular points. 
Take the edge cut by $\phi$ whose upper endpoint $x$ has minimum value, and let $e$
denote the portion above $\phi$. Then $\phi$ is L-homotopic to the contour
through any point in the interior of $e$. If $x$ is regular, then $\phi$ is L-homotopic
to the contour through $x$.
\end{claim}

\begin{proof} 
Let $a$ be the height of $\phi$ and $b$ the height of $x$. Consider the (potentially disconnected) 
manifold obtained by removing all manifold points above $b$ and below $a$.  
Specifically consider connected component that contains $\phi$.  
By continuity of L-homotopy the claim holds iff it holds on this manifold. 
However, this manifold contains no vectices except at the ends, and therefore
an L-homotopy can be constructed using the natural piecewise linear map.
\end{proof}

\subsection{Reeb graphs} \label{sec:reeb}

Now we get to the real definitions. Call a contour regular if it only contains regular points.

\begin{definition} \label{def:rel} Define a relation $\sim$ between regular contours, so $\phi \sim \phi'$ if $\phi$
is L-homotopic to $\phi$'.
\end{definition}

It is easy to check that $\sim$ is an equivalence relation, so the set of regular contours
is partitioned into equivalence classes. All contours in a class are L-homotopic to each other.

\begin{claim} \label{clm:equiv} Consider such an equivalence class $\Phi$. Then $f(\Phi)$
is an open interval $(a,b)$, where $a$ and $b$ are critical values.
\end{claim}

\begin{proof} \Clm{mono} and continuity of L-homotopy immediately imply that for $\phi, \phi' \in \Phi$,
for any $t \in [f(\phi),f(\phi')]$, there exists $\phi'' \in \Phi$ such that $f(\phi'') = t$.
So $f(\Phi)$ is an interval. As equivalence classes are defined over only regular contours, \Clm{reg} 
tells us that this interval must be open with critical value endpoints.
\end{proof}

We can now define the Reeb Graph $\reeb(\MM)$. We use the fact that critical points 
have distinct values. (Technically, the Reeb Graph depends on the function $f$.
Because it is implicit by the embedding of $\MM$, we will remove it from the notation.)

\begin{definition} \label{def:reeb} The \emph{Reeb Graph} $\reeb(\MM)$ has as a vertex set
the set of critical points. The edge $(x,y)$ is present if there exists equivalence class
$\Phi$ such that $f(\Phi) = (f(x),f(y))$.
\end{definition}

We define a ``cut" operation on manifolds that cuts along a contour to create
a manifold with a new boundary. Intuitively, any cut creates two disjoint boundaries, one
above and one below.

\medskip
\fbox{
\begin{minipage}{0.9\textwidth}
{\bf $\cut(\MM,\phi)$}

\smallskip
\begin{asparaenum}
%	\item Insert the vertices of $\phi$ in $\MM$.
	\item Take an arbitrary edge $e$ whose interior intersect $\phi$.
	\item Take a point $x$ (resp. $y$) on $e$ at distance $\eps$ above (resp. below) $\phi$.
	\item Let $\phi_x$ be the contour through $x$, and analogously define $\phi_y$. Insert a vertex
	at each intersection point of these contours with edges of the manifold. Triangulate all new faces.
	\item Delete the vertices (and incident edges, faces, etc.) of $\phi$ from the manifold.
\end{asparaenum}
\end{minipage}}

\medskip
This results in a new manifold which is boundary critical if the input manifold was. There are two new disjoint
boundary faces added. The manifold may be disconnected. Abusing notation, we use $\cut(\MM,\Psi)$
for a set $\Psi$ of contours to mean the manifold obtained by cutting along all contours in $\Psi$.

We now describe a procedure that converts the Reeb Graph of $\cut(\MM,\phi)$ to the Reeb Graph of $\MM$.

\medskip
\fbox{
\begin{minipage}{0.9\textwidth}
{\bf $\surgery(\MM,\phi)$}

\smallskip
\begin{asparaenum}
	\item Let $\MM' = \cut(\MM,\phi)$.
	\item Construct $\reeb(\MM')$ and let $B, C$ be the vertices corresponding to the new boundaries created
	in $\MM'$. (One is a minima and the other is maxima.)
	\item Since $B, C$ are leaves, they each have unique neighbors $B'$ and $C'$, respectively. Insert
	edge $(B',C')$ and delete $B,C$ to obtain a new graph.
\end{asparaenum}
\end{minipage}}

\medskip
\begin{lemma} \label{lem:surgery} For any regular contour $\phi$, the output of $\surgery(\MM,\phi)$ is $\reeb(\MM)$.
\end{lemma}

\begin{proof} The only difference between $\reeb(\MM)$ and $\reeb(\cut(\MM,\phi))$ is in the equivalence
class corresponding to $\phi$. This class is split into two classes in $\cut(\MM,\phi)$, which are joined
back by $\surgery(\MM,\phi)$.
\end{proof}

\section{Raining to partition $\MM$} \label{sec:rain}

In this section, we assume that $\MM$ is homeomorphic to $\SSS^2$ with boundaries. 

\begin{definition} \label{def:dom} A manifold is \emph{extremum dominant} if there exists
a maximum (resp. minimum) $x$ such that every non-maximal (resp. non-minimal) \emph{vertex} in the manifold has an increasing (resp. decreasing) path to $x$.
\end{definition}

We will describe a linear time algorithm that partitions $\MM$ into a set of extremum dominant manifolds.
The first part of this requires a ``raining" procedure. Start at some point $x \in \MM$ and imagine rain at $x$.
The water will flow downwards along descending paths and ``wet" all the points encountered.

\begin{definition} \label{def:wet} The set of points $y$ such that there is a descending path from $x$ to $y$
is denoted by $\wet(x,\MM)$. A point $z$ is on the \emph{interface} of $\wet(x,\MM)$ if every neighborhood of $z$
has non-trivial intersection with $\wet(x,\MM)$.
\end{definition}

Note that $\wet(x,\MM)$ (and its interface) can easily be computed in linear time in the size of the wet sub-manifold, using a descending BFS from $x$.

\begin{claim} \label{clm:inter} For any $x$, the interface of $\wet(x,\MM)$ is a set of disjoint, non-regular contours.
Furthermore, each such contour contains a merge vertex.
\end{claim}

\begin{proof}
 First observe that any point which in the interface must be wet.  Second, for $y \in \wet(x,\MM)$, all the points in any contour containing $y$ are also in $\wet(x,\MM)$, 
 as one can follow the descending path to $y$ and then walk along the contour.
 
 So let $y$ be a point in the interface.
 For $\eps$ sufficiently small, the points in any $\eps$ neighborhood of $y$ that lie below $y$ must have a descending path from $y$ and hence must be wet.
 Therefore in order for $y$ to have non-trivial intersection with $\wet(x,\MM)$ for any arbitrarily small $\eps$ there must be a dry point which lies 
 strictly above $y$.  In particular, this implies there must be an entire dry contour which lies just above $y$.
 
 However, since $y$ is in the interface, it is also wet.  Therefore for any contour $\phi$ which contains $y$, for any arbitrarily small $\eps$ there is 
 a wet point which lies above $\phi$ and has a descending path of length at most $\eps$ to $\phi$.
 This implies there is some contour which lies just 
 above $y$ which is wet.  So any contour $\phi$ containing $y$ has two contours which lie just above it, one which is dry, and one which is wet.  
 This can only happen if $\phi$ contains a merge vertex, because otherwise all contours which lie just above would be in the same L-homotopy equivalence class and so by 
 \Clm{mono}, if one is wet, all below it are wet.
 
 By similar logic one can argue that of the two contours through a given merge vertex at most one can contain interface points other than the merge vertex, 
 and that if one of the contours does contain such interface points, then the entire contour consists of interface points.
\end{proof}

% \begin{proof} If $y \in \wet(x,\MM)$, but not in the interface, then any contour containing $y$ is also in $\wet(x,\MM)$.
% So consider a point $y$ in the interface. We argue that all points in a contour $\phi$ containing $y$
% are also in the interface. If such a contour is regular, we can find a contour just above it that
% is also wet. 
% 
% If two contours intersect, the intersection point $y$ is critical. We can argue that the only
% way to reach $y$ from a descending path is reach one of the contours and then follow it to get to $y$.
% But that would imply the existence of another critical point on this contour, violating the Morse condition.
% 
% Let us argue that $y$ is a merge vertex. Consider the descending path from $x$ to $y$. The final edge
% that reaches $y$ is obviously all wet, and is an increasing edge from $y$. Consider an opposite increasing
% edge. Take two point in these edges that are infinitesimally higher than $y$ and show that the contours
% through them as disjoint. So $y$ is a merge.
% \end{proof}

We will define a \lift{} operation on the interface contours. Consider such a contour $\phi$ containing
a merge vertex $y$. Take any dry increasing edge incident to $y$, and pick the point $z$ on this edge at height
$f(y) + \eps$. Let $\lift(y) = \cont(z)$.

\begin{claim} \label{clm:cut-int} Let $\phi$ be an interface contour. Then $\cut(\MM,\lift(\phi))$
results in two disjoint manifolds, one consisting entirely of dry points.
\end{claim}

\begin{proof} Note that $\lift(\phi)$ is a closed, dry curve. By the Jordan Curve Theorem,
$\cut(\MM,\lift(\phi))$ results in two disjoint manifolds. Let $\NN$ be the manifold containing
the point $x$ that we called $\wet(x,\MM)$ from, and let $\NN'$ be the other manifold. 
If there exists a wet point $z$ in $\NN'$, then there exists a descending path from $x$ to $z$. 
As $x$ is in $\NN$ and $z$ is in $\NN'$, again by the Jordan Curve Theorem this path must intersect $\phi$, 
contradicting the fact that $\phi$ is dry.
\end{proof}

We now define the main partitioning procedure that cuts up a manifold $\NN$ homeomorphic to $\SSS^2$ with boundaries into a series of extremum
dominant manifolds. It takes as input the manifold and a maximum $x$. To initialize,
we begin with $\NN$ set to $\MM$ and $x$ as an arbitrary maximum.

\medskip
\fbox{
\begin{minipage}{0.9\textwidth}
{\bf $\rain(x,\NN)$}

\smallskip
\begin{compactenum}
	\item Compute the interface of $\wet(x,\NN)$. 
	\item If the interface is empty, simply output $\NN$. Otherwise, denote the contours by $\phi_1, \phi_2, \ldots, \phi_k$, and set $\phi'_i = \lift(\phi_i)$.
	\item Initialize $\NN_1 = \NN$.
	\item For $i$ from $1$ to $k$: 
	\begin{compactenum}
		\item Call $\cut(\phi'_i,\NN_1)$ to get the dry manifold $\LL_i$ and wet manifold $\NN_{i+1}$.
		\item Let the new boundary of $\LL_i$ be $B_i$. Invert $\LL_i$ so that $B_i$ is a maximum. Recursively
		call $\rain(B_i,\LL_i)$.
	\end{compactenum}
	\item Output $\NN_{k+1}$.
\end{compactenum}
\end{minipage}}

\medskip
For convenience, denote the total output of $\rain(x,\MM)$ by $\MM_1, \MM_2, \ldots, \MM_r$.

\begin{lemma} \label{lem:rain-1} Each output manifold $\MM_i$ is extremum dominant.
\end{lemma}

\begin{proof} Consider a call to $\rain(x,\NN)$. If the interface is empty, then all of $\NN$
is in $\wet(x,\NN)$, so $\NN$ is trivially extremum dominant. So suppose the interface
is non-empty and consists of $\phi_1, \phi_2, \ldots, \phi_k$ (as denoted in the procedure).
By repeated applications of \Clm{cut-int}, $\NN_{k+1}$ consists of wet points. 
Consider $\wet(x,\NN_{k+1})$. The interface must exactly be $\phi_1, \phi_2, \ldots, \phi_k$.
So the only dry vertices are those in the boundaries $B_1, B_2, \ldots, B_k$. But these
boundaries are maxima.
\end{proof}

The procedure $\rain$ creates new boundary vertices on the edges it cuts each time it call the 
subroutine $\cut$.  The key to understanding the running time of this procedure is to bound the 
number of newly created vertices, for which we have the following lemma.

\begin{lemma}\label{lem:new-verts}
The total number of boundary vertices created by a call to $\rain(x, \NN)$ is $O(n)$, where $n=|\NN|$
\end{lemma}
\begin{proof}
As the number of edges is $O(n)$, arguing that there is at most one boundary vertex created along any given edge will imply the claim.
Consider some edge $e$ and take the first time a boundary vertex is created in $e$.
This happens when some contour $\phi$ intersects $e$. On applying $\cut(\NN,\lift(\phi))$,
consider the dry manifold $\LL$ and wet manifold $\MM$ constructed. The edge $e$ is split into two parts, the dry and wet part.
The wet part is in $\MM$ and will never be involved in any recursive call.

The dry part of $e$ (call it $e'$) needs to be considered.
A boundary vertex $v$ (an new endpoint of $e$) is created, which is part of the boundary $B$ in $\LL$,
such that $\rain(B,\LL)$ is called to the inverted $\LL$. In this call, since the water starts from $v$,
all of $e'$ becomes wet. So there is no further interface contour that can intersect this portion of $e'$.
\end{proof}

As each original vertex in $\NN$ appears in only one of the sub-manifolds output by $\rain(x,\NN)$, we have the following corollary.

\begin{corollary}
The total size of a sub-manifolds output by $\rain(x,\MM)$ is $O(n)$.
\end{corollary}

\begin{corollary} \label{cor:rain-time} The total running time of $\rain(x,\MM)$ is $O(n)$.
\end{corollary}

\begin{proof} Consider some invocation to $\rain(x,\MM)$. The only non-trivial work done in $\rain$ is by the calls to $\wet$.
We know that each invocation to $\wet$ runs in time linear 
to the portion of the manifold which is wet (including the new boundary vertices created).  
As the output of $\rain$ is precisely the distinct sub-manifolds wet by each distinct call to $\wet$, the claim follows.
\end{proof}

Consider the set of manifolds output by $\rain(x,\MM)$, there is a natural tree ordering defined over these manifolds where the 
root corresponds the first manifold that was wet when we rained from $x$, and its children correspond to the recursive calls 
(i.e. where we rained from the interface contours of the first raining).  Call this the \emph{Manifold Ordering Tree}.
$\rain$ can easily be modified to also output this tree (without affecting the asymptotic running time).  

%Consider one of the manifolds $\MM_i$ output by $\rain$.  This manifold has a set of boundary maxima/minima that are the result 
%of cutting along interface contours.  For each such contour there is exactly one other output manifold $\MM_j$ for which 
%the contour appears as a boundary maxima/minima. 

\begin{claim} \label{clm:rain-reeb} Let $\reeb(\MM_1), \reeb(\MM_2), \ldots, \reeb(\MM_r)$, 
be the output of $\rain(x,\MM)$. Suppose that for each reeb graph we know its corresponding vertex in the Manifold Ordering Tree. 
Then $\reeb(\MM)$ can be computed in $O(n)$ time.
\end{claim}

\begin{proof}
We will walk through the Manifold Ordering Tree in a leaf first ordering.  Each time we visit a node we connect its reeb graph to the 
reeb graph of its children in the tree using the $\surgery$ procedure. This takes $O(n)$ time overall by \Lem{new-verts} and since $\surgery$ takes constant time (given the appropriate vertices in each reeb graph).  The correctness follows by induction and  \Lem{surgery}.
\end{proof}

\section{Painting to compute contour trees} \label{sec:paint}
The previous section allows us to restrict attention to extremum dominant manifolds.
We will orient so that the extremum in question is a \emph{minimum}.
We will fix such a manifold $\MM$, with the dominant minimum $u$. 

We begin by associating a color with each maximum. If the set of maxima is $S$,
this is simply an arbitrary bijection $\chi:S \mapsto [|S|]$. Imagine there being a large
can of paint of color $\chi(y)$ at maxima $y$. We will spill different paint from each maximum and watch it flow down.
This is analogous to the raining in the previous section, but paint is a much more viscous liquid.
\emph{So paint only flows down edges, and it does not color the interior of faces.} Furthermore, paints
do not mix, so eventually, every vertex and edge of $\MM$ gets a unique color.
%
\begin{definition} \label{def:paint} A  \emph{painting} of $\MM$ is a map $\chi:S \cup E \mapsto [|S|]$
with the following properties. (Remember that $G(\MM) = (V,E)$.)
\begin{asparaitem}
	\item The restriction $\chi:S \mapsto [|S|]$ is a bijection.
	\item Consider an edge $e$. There exists a descending path from some maximum $x$ to $e$
	consisting of edges in $E$, such that all edges along this path have the same color as $x$. (The maximum $x$
	is well-defined to be the unique maximum with the same color as $e$.)
\end{asparaitem}
\end{definition}
%
Note that paintings are not unique. It is straightforward to construct a painting of $\MM$ in $O(n)$ time, using a descending BFS from each maximum that does not explore previously colored edges. 

\subsection{The data structures} \label{sec:struct}

\begin{definition} \label{def:color-set} Fix a painting $\chi$, and a color $c$.
A critical point $x$ is \emph{touched by $c$} if $x$ is incident to a $c$-colored
increasing edge. The set of non-extremal critical points touched by $c$ is denoted $\touch(c)$.
For $x$, $\col(x)$ is the set of colors that touch $x$.
\end{definition}
%
%We will store each set $\touch(c)$ in a binomial heap, keyed by the heights. 
Abusing notation, $\touch(c)$ refers
both to the set and the data structure used to store it.

\medskip
\noindent
{\bf The binomial heaps $\touch(c)$:} For each color $c$, we store a binomial
heap $T(c)$ of non-extremal critical points, keyed by vertex heights.

\medskip
\noindent
{\bf The union-find data structure on colors:} We will repeatedly perform unions
of classes of colors, and this will be maintained as a standard union-find data structure.
For any color $c$, $\rep(c)$ denotes the representative of its class. We initialize
by simply setting $\rep(c) = c$ for all colors.


\medskip
\noindent
{\bf The stack $\st$:} This consists of non-extremal critical points with the following structure.
Each point $x \in \st$ has an associated subset of $\col(x)$, denoted $\mcol(x)$.
Both $\mcol(x)$ and its complement are stored as hash table. So lookups, inserts, and deletes
are in these sets are all constant time operations. The stack is guaranteed to satisfy 
the following invariants.
\begin{asparaitem}
	\item For every $x \in \st$: For every $c \in \mcol(x)$, $x$ is the highest element
	in $T(c)$. Furthermore, $c = \rep(c)$.
	\item Consider $x, y \in \st$ such that $y$ was pushed on $x$. There exists $c \in \col(x) \setminus
	\mcol(x)$ such that $x$ is not highest in $T(c)$ but $y$ is highest in $T(c)$.
\end{asparaitem}

\medskip
\noindent
{\bf Highest points $\h(c)$:} For each color $c$, we maintain a special critical point $\h(c)$ of this color.
It is supposed to denote the ``highest" point of color $c$, but the notion of highest evolves
with the algorithm. 

\subsection{Updating the stack} \label{sec:stack}

We have a simple, but critical, procedure that updates the stack $\st$ by pushing
points on to it.


\medskip
\fbox{
\begin{minipage}{0.9\textwidth}
{\bf $\update(\st)$}

\smallskip
\begin{compactenum}
	\item If $K$ is empty, pick arbitrary $c$. Let $x$ be the highest point in $T(c)$.
	Push $x$ onto stack and set $\mcol(x) = \{c\}$.
	\item Let $h$ be head of $\st$.
	\item While $\col(h) \setminus \mcol(h)$ non-empty:
	\begin{compactenum}
		\item Pick some $c \in \col(h) \setminus \mcol(h)$. Replace it by $c' = \rep(c)$.
		\item Obtain highest point $x$ in $T(c')$. 
		\item If $x = h$, put $c'$ in $\mcol(h)$. Otherwise ($x \neq h$), push $x$ onto stack
		and add $c'$ to $\mcol(x)$.
		\item Let $h$ be new head of $\st$.
	\end{compactenum}
\end{compactenum}
\end{minipage}}

\medskip
When $\update(\st)$ terminates, the head $h$ has the following property. For every $c \in \col(h)$,
$h$ is the highest point in $\touch(c)$. 

\begin{claim} \label{clm:cont} Consider an increasing path from $h$ to some maxima $x$.
For any $y$ in this path, $\cont(y)$ only contains points of color $c(x)$.
Furthermore, for all vertices on $\cont(y)$, their increasing edges are colored $c(x)$.
\end{claim}

\begin{proof} Consider this increasing path, and think of moving $y$ continuously downwards from $x$
to $h$. Initially (when $y = x$), the lemma is trivially true. Consider the first $y$
when the condition fails. We have $\cont(y)$ only containing points of color $c(x)$,
but there is a vertex $z \in \cont(y)$ that has increasing edges of a different color $c'$.
All the increasing edges of $z$ that are colored $c(x)$ are contiguous (since we can construct
a contour intersecting all of them). Thus, there are two disjoint increasing neighborhoods for $z$,
and $z$ is a saddle point. But that contradicts the fact that $h$ is the highest saddle point
of color $c(x)$.

So, as $y$ decreases, for all vertices on $\cont(y)$, their increasing edges are colored $c(x)$.
This also means that $\cont(y)$ only contains points of color $c(x)$.
\end{proof}

We have important lemmas characterizing $|\col(h)|$.

\begin{lemma} \label{lem:mono} If $h$ is a merge vertex, then $|\col(h)| > 1$.
\end{lemma}

\begin{proof} Suppose not. Consider the edges incident to $h$ in circular order. There exist (at least) $2$
disjoint contiguous stretches of edges that are increasing. 
All these edges have the same color $c$, associated with some maximum $x$. Consider two neighbors $y,z$
that lie in different stretches. By the definition of a painting, there exist descending paths (along
edges) from $x$ to $y$ and $x$ to $z$. We can construct a Jordan curve $C$ from $x \rightarrow y
\rightarrow h \rightarrow z \rightarrow x$. All points in this curve except $h$ have 
heights larger than $f(h)$. By the Jordan Curve Theorem, this partitions $\MM$ into
an ``inside" and an ``outside". 

One of the lower neighbors of $h$ (say $p$) is inside, and another (say $q$) is outside.
Because $h$ is a merge, neither of these can a minimum other than the dominant minimum $u$.
Hence, both $p$ and $q$ have descending paths to $u$. So there is a path from $p$
to $q$ where all heights are strictly less than $f(h)$. But this path must intersect $C$.
Contradiction.
\end{proof}

\begin{lemma} \label{lem:col} If $h$ is a merge vertex, $|\col(h)| = 2$.
If $h$ is a split vertex, $|\col(h)| = 1$.
\end{lemma}

\begin{proof} Consider one contiguous sequence of increasing edges from $h$.
There exists a contour intersecting all of these edges. By \Clm{cont}, they all
must have the same color. There are two such contiguous sequences.

If $h$ is a merge vertex, then $|\col(h)| = 2$ by \Lem{mono}. If $h$ is a split vertex,
there is a single contour passing through all increasing edges. By \Clm{cont}, they
all have the same color.
\end{proof}

\subsection{Constructing $\reeb(\MM)$} \label{sec:const}

The main construction procedure is $\onestep(\MM)$. It pieces together $\reeb(\MM)$
incrementally. 

\medskip
\fbox{
\begin{minipage}{0.9\textwidth}
{\bf $\onestep(\MM)$}

\smallskip
\begin{compactenum}
	\item Run $\update(\st)$. Pop $\st$ to get $h$.
	\item Suppose $\mcol(h) = \{c\}$ (so $h$ is split). 
	\begin{compactenum}
		\item Connect $h$ (in $\reeb(\MM)$) to $\h(c)$ and the unique minimum corresponding to split $h$. 
		\item Delete $h$ from $T(c)$, set $\h(c) = h$.
		\item End procedure.
	\end{compactenum}
	\item Let $\mcol(h) = \{c_1, c_2\}$ (so $h$ is merge).
	\item Connect $h$ in $\reeb(\MM)$ to $\h(c_1)$ and $\h(c_2)$.
	\item Delete all copies of $h$ from $T(c_1)$ and $T(c_2)$.
	\item Perform union of colors $c_1$ and $c_2$ (denote merged color as $c$). Merge heaps
	to get $T(c) = T(c_1) \cup T(c_2)$.
	\item Set $\h(c) = h$.	
\end{compactenum}
\end{minipage}}

\medskip
\begin{theorem} \label{thm:piece} Consider the very first call to $\onestep(\MM)$. Let $y$ be a point
infinitesimally below $h$ on a decreasing edge. Consider $\cut(\MM,\cont(y)) = \{\MM_u,\MM_\ell\}$ (the upper and lower manifolds). Then $\onestep(\MM)$
outputs $\reeb(\MM_u)$ and implicitly gives a valid painting of $\MM_\ell$.
\end{theorem}

We have an important lemma that helps to bound the running time.

\begin{lemma} \label{lem:tree} For every color $c$, all points in $T(c)$ lie along a monotone path in 
$\reeb(\MM)$.
\end{lemma}

\begin{proof} Prove by induction of the number of edges in $\reeb(\MM)$. Consider the head $h$ of $\st$ in the first call to $\onestep(\MM)$,
and the contour $\cont(y)$ as defined in \Thm{piece}. Take $\cut(\MM,\cont(y)) = \{\MM_u,\MM_\ell\}$. Apply induction
hypothesis on $\MM_\ell$, and use that to argue about $T(c_1)$ and $T(c_2)$ in the original manifold.
\end{proof}

\end{document}
